%% start of file `template.tex'.
%% Copyright 2006-2010 Xavier Danaux (xdanaux@gmail.com).
%
% This work may be distributed and/or modified under the
% conditions of the LaTeX Project Public License version 1.3c,
% available at http://www.latex-project.org/lppl/.
\documentclass[10pt, a4paper]{moderncv}

\moderncvtheme[grey, roman]{classic}

\usepackage{ifxetex}
\ifxetex % only for xelatex
\usepackage{xltxtra} % this should load both fontspec & xunicode
\usepackage{fontspec}
\setromanfont{Helvetica Neue}
\else %
\usepackage[utf8]{inputenc}
\usepackage[T1]{fontenc}
\fi % \ifxetex

% adjust the page margins
\usepackage[scale=0.8]{geometry}
\setlength{\hintscolumnwidth}{3cm}						% if you want to change the width of the column with the dates
\AtBeginDocument{\setlength{\maketitlenamewidth}{6cm}}  % only for the classic theme, if you want to change the width of your name placeholder (to leave more space for your address details
\AtBeginDocument{\recomputelengths}                     % required when changes are made to page layout lengths

%Publications section rename
\renewcommand{\refname}{Publications}

% personal data
\firstname{Nicolas}
\familyname{Goles}
%\title{Software Engineering Student}               % optional, remove the line if not wanted
\address{}{Santiago, Chile}    % optional, remove the line if not wanted
\email{nicolas@gandogames.com}                      % optional, remove the line if not wanted
\homepage%
{%
	http://gandogames.com
}

%	\fontspec[Ligatures={Common, Historical}]{Helvetica Neue}
%	\fontsize{10pt}{18pt}\selectfont @ngoles

% to show numerical labels in the bibliography; only useful if you make citations in your resume
\makeatletter
\renewcommand*{\bibliographyitemlabel}{\@biblabel{\arabic{enumiv}}}
\makeatother

%\newcommand\Colorhref[3][cyan]{\href{#2}{\small\color{#1}#3}}
\usepackage[usenames,dvipsnames]{xcolor}
\definecolor{darkgray}{rgb}{0.4,0.4,0.4}
\newcommand\Colorhref[1]{\small\textcolor{RoyalBlue}{\url{#1}}}

%----------------------------------------------------------------------------------
%            content
%----------------------------------------------------------------------------------
\begin{document}
\maketitle

\begin{small}
\textcolor{darkgray}{
\textbf{I'm a strongly motivated team player with great social skills, working to improve my abilities on a daily basis. Always interested in exciting and challenging projects in creative environments where I can
both contribute and learn from a great team of people.}}
\end{small}

%----------------------------------------------------------------------------------
%            Skills
%----------------------------------------------------------------------------------
\section{Skills}
\cvline{Languages}{C++, C, Objective-C, Lua, Python}
\cvline{Technologies}{Cocoa Touch, Cocoa, OpenGL ES 1.1/2.0, OpenGL}
\cvline{Software}{Xcode, gcc, Git, svn, gdb, Bash, Vim}
\cvline{Idioms}{Bilingual, Spanish and English. (\textbf{108/120 TOEFL Score}), Basic French}


%----------------------------------------------------------------------------------
%            Education
%----------------------------------------------------------------------------------
\section{Education}
\cventry{\begin{small}Current\end{small}}{Software Engineer}{UTFSM}{Santiago, Chile}{last year student}{\begin{scriptsize}To graduate on 2012.\end{scriptsize}}  % arguments 3 to 6 can be left empty

\section{Experience}

\cventry{\begin{small}Current\end{small}}{CTO}{Apparel Dream}{Santiago, Chile}{} 
{ 
\begin{scriptsize} 
CTO of Apparel Dream, a 99Designs for Fashion startup based in Santiago, Chile. My main responsibility is to ensure the successful execution of the company’s business mission through development and
deployment of the company’s on-line presence. More specifically, my work involves managing a development team, using Ruby on Rails to architect and develop our Web platform and planning for risk and growth.\\
\end{scriptsize}
}

\cventry{\begin{small}Aug 2011--Nov 2011\end{small}}{Intern (Engineering)}{INRIA, Lognet Team}{Sophia Antipolis, France}{}
{
\begin{scriptsize}
Designed and developed the mobile Application and API's for myMed, a social platform being developed by France (INRIA, Sophia Antipolis), and Italy (University of Torino and Polytechnic of Torino).
Implemented a Binary \& Web hybrid client for iOS that provided a bridge between Javascript and Objective-C code in order to access the device hardware from an HTML page (Camera, Accelerometer,
Audio) presented in the binary.\\
\end{scriptsize}
}

\cventry{\begin{small}April 2010--Dec 2011\end{small}}{Lead Engineer}{Gando Games}{Santiago, Chile}{}
{
\begin{scriptsize}
Founder and Lead Engineer of Gando Games, an independent Game Development Studio based in Chile and focused on iOS game development. Released 2 games to Apple's App Store and did the whole development of Gando Games internal 2D Game Engine, which was fully scripted with Lua (around 30k lines of C++, C and Lua).\\
\end{scriptsize}
}

\cventry{\begin{small}Jan 2010--Mar 2010\end{small}}{Intern (R \& D)}{INRIA, Lognet Team}{Sophia Antipolis, France}{}
{
\begin{scriptsize}
Developed the first prototype of a multi-platform Peer-to-Peer engine (Static Library) to be used in desktop computers and mobile devices (iPhone OS, Mac OS X, Linux and Android OS). This involved C and C++ knowledge, Network Programming, concurrent programming and cross-compiling for several architectures among others.\\
\end{scriptsize}
}

\cventry{\begin{small}Oct 2008--Aug 2009\end{small}}{iOS Developer}{Baytex Software}{Santiago, Chile}{}
{
\begin{scriptsize}
Working in Game development for the iPhone Platform with the Cocoa API/Objective-C using cutting edge mobile technologies like OpenGL ES and Cocoa Touch API.\\
\end{scriptsize}
}

\cventry{\begin{small}Dec 2008--Jan 2009\end{small}}{Intern (Software Development)}{Synopsys Inc.}{Santiago, Chile}{}
{
\begin{scriptsize}
Internship as a developer at the PRESTO Compiler group at Synopsys office in Chile. Worked in AMD64 architecture porting and bug fixing of the Presto RTL (register transfer level) languages compiler at
PRESTO group for a multinational EDA (Electronic Design Automation) enterprise. \\
\end{scriptsize}
}

%----------------------------------------------------------------------------------
%            Projects & Interests
%----------------------------------------------------------------------------------
\section{Projects \& Interests}
\cventry{\begin{small}July 2009--Current\end{small}}{Stack Overflow}{}{}{}
{
\begin{small}Average top 15\% user, with a ~2.5k reputation.\end{small}\\
\Colorhref{http://stackoverflow.com/users/145077/mr-gando}\\
}
\cventry{\begin{small}May 2011--Current\end{small}}{Github}{Xcode 4 Template Generator}{}{}
{
\begin{small}Python script to help in the process of making Xcode 4 Templates.\end{small}\\
\Colorhref{ https://github.com/MrGando/Xcode-4-Template-Generator}\\
}
\cventry{\begin{small}15+ years\end{small}}{Piano Playing}{}{}{}
{
Have been studying piano nearly all my life, music is a huge part of me.
}

%----------------------------------------------------------------------------------
%            Awards
%----------------------------------------------------------------------------------
\section{Awards}
\cventry{\begin{small}September 2009\end{small}}{Contest}{National iPhone development contest winner}{}{}
{
\begin{scriptsize}
Winner of the Apple Inc. supported Chilean Applicate iPhone development contest with the game Hex Reaction, the game was chosen winner from a total of 150 participants. Earned conference entry ticket, airplane tickets and hotel fees for the Apple World Wide Developers Conference (WWDC) San Francisco 2010.\\
\end{scriptsize}
}

\cventry{\begin{small}October 2009\end{small}}{Academic}{Outstanding Student in extracurricular activities}{}{}
{
\begin{scriptsize}
Award given at Universidad T\'ecnica Federico Santa Mar\'ia, to students that distinguish themselves in personal activities or projects outside of the University that are related to their field of study.
\end{scriptsize}
}

%----------------------------------------------------------------------------------
%            Publications
%----------------------------------------------------------------------------------
\nocite{*}
\bibliographystyle{plain}
\bibliography{publications}       % 'publications' is the name of a BibTeX file

%----------------------------------------------------------------------------------
%            References
%----------------------------------------------------------------------------------
\section{References}
\cventry{}{Available upon Request}{}{}{}
{
}

\end{document}


%% end of file `template_en.tex'.
